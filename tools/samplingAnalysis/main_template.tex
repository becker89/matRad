% %%%%%%%%%%%%%%%%%%%%%%%%%%%%%%%%%%%%%%%%%%%%%%%%%%%%%%%%%%%%%%%%%%%%%%%%%
% matRad uncertainty analysis report latex template
% 
% call
%   e.g. 'xelatex  main.tex' in console
% output
%   pdf report
%
%
% %%%%%%%%%%%%%%%%%%%%%%%%%%%%%%%%%%%%%%%%%%%%%%%%%%%%%%%%%%%%%%%%%%%%%%%%%

% %%%%%%%%%%%%%%%%%%%%%%%%%%%%%%%%%%%%%%%%%%%%%%%%%%%%%%%%%%%%%%%%%%%%%%%%%
%
% Copyright 2017 the matRad development team. 
% 
% This file is part of the matRad project. It is subject to the license 
% terms in the LICENSE file found in the top-level directory of this 
% distribution and at https://github.com/e0404/matRad/LICENSES.txt. No part 
% of the matRad project, including this file, may be copied, modified, 
% propagated, or distributed except according to the terms contained in the 
% LICENSE file.
%
% %%%%%%%%%%%%%%%%%%%%%%%%%%%%%%%%%%%%%%%%%%%%%%%%%%%%%%%%%%%%%%%%%%%%%%%%%
\documentclass[a4paper]{scrartcl}
\usepackage[english]{babel}
\usepackage[utf8]{inputenc}
\usepackage{amsmath}
\usepackage{graphicx}
\usepackage[autoplay,loop]{animate}
\usepackage[verbose]{placeins}
\usepackage{fullpage}
\usepackage{caption}
\usepackage{subcaption}
\usepackage[colorinlistoftodos]{todonotes}
\addtokomafont{disposition}{\rmfamily}

% pgfplots for matlab2tikz
\usepackage{pgfplots}
\pgfplotsset{compat=newest}
%% the following commands are needed for some matlab2tikz features
\usetikzlibrary{plotmarks}
\usetikzlibrary{arrows.meta}
\usepgfplotslibrary{patchplots}
\usepackage{grffile}
\usepackage{amsmath}
\usepackage{booktabs}
\usepackage{color}
\usepackage{fancyhdr}
\usepackage{footnote}
\usepackage[hidelinks]{hyperref} % clickable table of content
\makesavenoteenv{figure}

\usetikzlibrary{external}
\tikzexternalize[prefix=data/figures/]

\newlength\figW
\setlength{\figW}{0.9\textwidth}
\newlength\figH
\setlength{\figH}{0.56\textwidth} % 0.9 / golden ratio

\usepackage[babel,english=british]{csquotes}


\begin{document}
\input{data/parameters.tex}
\input{data/patientInformation.tex}

\title{Uncertainty Sampling Report}
\subtitle{\patientLastName{}, \patientFirstName{}, \patientID\\ \textcolor{red}{\warnMessage}}
\author{\operator}
% \date{\today}
\maketitle
\begin{figure}[!b]
  \centering
  \animategraphics[controls,width = \textwidth]{\framerate}{data/frames/anim_}{\firstframe}{\lastframe}
\end{figure}

\newpage
\tableofcontents

\newpage
\section{Patient Information}
\label{sec:patientInformation}

\begin{table}[h]
\centering
%\caption{My caption}
\label{table:patientInformation}
\begin{tabular}{rl}
\toprule
Patient's name & \patientLastName{}, \patientFirstName \\
Patient ID     & \patientID                             \\
Patient's sex  & \patientSex \\
\bottomrule
\end{tabular}
\end{table}

\section{Plan Overview}
\label{sex:planOverview}

\subsection{Setup}

\begin{table}[h]
  \centering
  %\caption{My caption}
  \label{table:planInformation}
  \begin{tabular}{rl}
    \toprule
    Gantry angles   & \planGantryAngles \\
    Couch angles    & \planCouchAngles  \\
    Radiation modalitiy & \planRadiationModality \\
    \bottomrule
  \end{tabular}
\end{table}

\subsection{Parameter for Uncertainty Analysis}
\input{data/uncertaintyParameters.tex}
The following parameters were used as scenarios, set by the user \operator.
\begin{table}[!h]
  \centering
  %\caption{My caption}
  \label{table:userParameters}
  \begin{tabular}{rl}
    \toprule
    Number of shifts       & \numOfShiftScen \\
    Shift size             & \shiftSize  \\
    Shift combination     & \shiftCombType  \\
    Shift isotropy         & \shiftGenIsotropy  \\
    Number of range shifts & \numOfRangeShiftScen \\
    Max abs. range Shift   & \maxAbsRangeShift  \\
    Max rel. range Shift   & \maxRelRangeShift  \\
    Range shift combination & \rangeCombType  \\
    Total Scen combination & \scenCombType  \\
    \bottomrule
  \end{tabular}
\end{table}

The following uncertainties were incorporated in this analysis.
\begin{table}[!h]
  \centering
  %\caption{My caption}
  \label{table:typeUncertainty}
  \begin{tabular}{rl}
    \toprule
    Ct uncertainties   & \ctScen \\
    Shift uncertainties    & \shiftScen  \\
    Range uncertainties & \rangeScen \\
    \bottomrule
  \end{tabular}
\end{table}

Assuming the uncertainties above to be distributed by a normally around 0 with standard deviaton
\begin{table}[!h]
  \centering
  %\caption{My caption}
  \label{table:uncertaintySD}
  \begin{tabular}{rl}
    \toprule
    $\sigma_{shift} $    & \shiftSD  \ mm\\
    $\sigma_{relRange} $ & \rangeRelSD \ \% \\
    $\sigma_{absRange} $ & \rangeAbsSD \ mm \\
    \bottomrule
  \end{tabular}
\end{table}

\subsection{Explenatory Information}
In section \ref{sec:usa} common quality indictors are used to examine their behaviour under the influence of uncertainties. 
The quality indicator can also be found in the table (top row, column-wise).
Metrics applied on an individual quality indicator due to the scenarios can be found in the first column (row-wise).

\subsubsection{Statistics}
All types of uncertainties are assumed to be uncorrelated, i.e. range uncertainties are not correlated to shift uncertainties.
Shift uncertainties are assumed to be ineherently uncorrelated as well, i.e. no correlation between shifts in x-, y- or z direction.
Therefore the probability assigned to a scenario $i$ is the product of the probabilities of the individual components $j$, i.e.
\begin{equation}
  p_i = \prod\limits_j p_j.
\end{equation}
 
\paragraph{Mean} $\mu$ on scenarios $X_i$ is weighted by the probabilities $p_i$:
\begin{equation}
  \mu = \frac{\sum\limits_{i} p_i X_i}{\sum\limits_{i} p_i}
\end{equation}
 
\paragraph{Variance} is analogously defined as:
\begin{equation}
	\sigma^2 = \frac{\sum\limits_{i} p_{i}\left(x_{i}-\mu\right)^{2}}{\sum\limits _{i}p_{i}}
\end{equation}

\subsubsection{Parameters}
The complete explaination of parameters can be found next to the usage in every \textit{setupStudy.m}.
\paragraph{shiftGenIsotropy} is set by default to \textit{+-}, which means that shifts go from negative to positive direction.
One can also choose to only use positive or negative shifts.
\paragraph{Combination Type} is a paremeter which is used for shift scenarios, range scenarios and the total combination of both (i.e. shiftCombType, rangeCombType, scenCombType).
It can be set to \textit{individual}, where one type is computed after the other, \textit{combined} combines both types directly and \textit{permuted} is only available for shifts, where every possible permutation of the given x-, y- and z-shifts is being sampled.






\FloatBarrier
\newpage

\section{Nominal Plan}
\subsection{Slices at isoCentre}
\begin{figure}[!b]
  \centering
  \input{data/isoSlicePlane1_nominal.tex}
  \caption{CT and nominal dose at isocentre. Coronal view.}
\end{figure}

\begin{figure}[!b]
  \centering
  \input{data/isoSlicePlane1_stdW.tex}
  \caption{CT and standard deviation of dose at isocentre. Coronal view.}
\end{figure}

\begin{figure}[!b]
  \centering
  \input{data/isoSlicePlane1_gamma.tex}
  \caption{CT and $\gamma$ index of nominal and mean dose (2\%, 2mm, global). Coronal view.}
\end{figure}

\begin{figure}[!b]
  \centering
  \input{data/isoSlicePlane2_nominal.tex}
  \caption{CT and nominal dose at isocentre. Saggital view.}
\end{figure}

\begin{figure}[!b]
  \centering
  \input{data/isoSlicePlane2_stdW.tex}
  \caption{CT and standard deviation of dose at isocentre. Saggital view.}
\end{figure}

\begin{figure}[!b]
  \centering
  \input{data/isoSlicePlane2_gamma.tex}
  \caption{CT and $\gamma$ index of nominal and mean dose (2\%, 2mm, global). Saggital view.}
\end{figure}

\begin{figure}[!b]
  \centering
  \input{data/isoSlicePlane3_nominal.tex}
  \caption{CT and nominal dose at isocentre. Axial view.}
\end{figure}

\begin{figure}[!b]
  \centering
  \input{data/isoSlicePlane3_stdW.tex}
  \caption{CT and standard deviation of dose at isocentre. Axial view.}
\end{figure}

\begin{figure}[!b]
  \centering
  \input{data/isoSlicePlane3_gamma.tex}
  \caption{CT and $\gamma$ index of nominal and mean dose (2\%, 2mm, global). Axial view.}
\end{figure}


\FloatBarrier

\newpage
\subsection{DVH and Quality Indicator}
\begin{center}
\input{data/nominalDVH.tex}
\end{center}
\input{data/nominalQI.tex}

\FloatBarrier
\newpage
\section{Uncertainty Scenario Analysis}
\label{sec:usa}

\input{data/structureWrapper.tex}




\end{document}
